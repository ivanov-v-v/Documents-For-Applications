%%%%%%%%%%%%%%%%%%%%%%%%%%%%%%%%%%%%%%%%%%%%%%%%%%%%%%%%%%%%%%%%%%%%%%%%%%%%%%%%
% Medium Length Graduate Curriculum Vitae
% LaTeX Template
% Version 1.2 (3/28/15)
%
% This template has been downloaded from:
% http://www.LaTeXTemplates.com
%
% Original author:
% Rensselaer Polytechnic Institute 
% (http://www.rpi.edu/dept/arc/training/latex/resumes/)
%
% Modified by:
% Daniel L Marks <xleafr@gmail.com> 3/28/2015
%
% Important note:
% This template requires the res.cls file to be in the same directory as the
% .tex file. The res.cls file provides the resume style used for structuring the
% document.
%
%%%%%%%%%%%%%%%%%%%%%%%%%%%%%%%%%%%%%%%%%%%%%%%%%%%%%%%%%%%%%%%%%%%%%%%%%%%%%%%%

%-------------------------------------------------------------------------------
%	PACKAGES AND OTHER DOCUMENT CONFIGURATIONS
%-------------------------------------------------------------------------------

%%%%%%%%%%%%%%%%%%%%%%%%%%%%%%%%%%%%%%%%%%%%%%%%%%%%%%%%%%%%%%%%%%%%%%%%%%%%%%%%
% You can have multiple style options the legal options ones are:
%
%   centered:	the name and address are centered at the top of the page 
%				(default)
%
%   line:		the name is the left with a horizontal line then the address to
%				the right
%
%   overlapped:	the section titles overlap the body text (default)
%
%   margin:		the section titles are to the left of the body text
%		
%   11pt:		use 11 point fonts instead of 10 point fonts
%
%   12pt:		use 12 point fonts instead of 10 point fonts
%
%%%%%%%%%%%%%%%%%%%%%%%%%%%%%%%%%%%%%%%%%%%%%%%%%%%%%%%%%%%%%%%%%%%%%%%%%%%%%%%%

\documentclass[11pt]{res}  
%\usepackage[a4paper, left=20mm, right=20mm, top=30mm, bottom=20mm]{geometry}
%\usepackage{amsmath,amsfonts,amssymb,amsthm,epsfig,epstopdf,titling,url,array,tkz-berge}
%\usepackage[T2A,T1]{fontenc}
\usepackage[utf8]{inputenc}
\usepackage[russian]{babel}
%\usepackage{xparse}
%\usepackage[shortlabels]{enumitem}
%\usepackage{float}
%\usepackage{cancel}
%\usepackage{multicol}
\usepackage{makecell}

\usepackage{hyperref}
\hypersetup{
	colorlinks=true,
	linkcolor=blue,
	filecolor=magenta,      
	urlcolor=blue,
}

% Default font is the helvetica postscript font
\usepackage{helvet}

% Increase text height
\textheight=800pt

\newcommand{\vmarginsmall}{\vspace{0.1cm}}
\newcommand{\vmargin}{\vspace{0.3cm}}

\begin{document}

%-------------------------------------------------------------------------------
%	NAME AND ADDRESS SECTION
%-------------------------------------------------------------------------------
%\name{Viacheslav Ivanov}

% Note that addresses can be used for other contact information:
% -phone numbers
% -email addresses
% -linked-in profile
%\name{Viacheslav Ivanov}

% Мне не нравится, что имя нельзя выровнять по высоте
\address{+7 (916) 564-22-94\\141701, Dolgoprudny,\\ Moscow Region, Russia}
\address{\huge{\textbf{Viacheslav Ivanov}}\\\qquad\qquad\quad\ 05/28/1997\\}
\address{\href{mailto://ivanov.vv@phystech.edu}{ivanov.vv@phystech.edu}\\\href{https://github.com/ivanov-v-v }{github:ivanov-v-v}\\\href{https://www.linkedin.com/in/vvivanov/ }{linkedin:vvivanov}}

%-------------------------------------------------------------------------------

\begin{resume}
%-------------------------------------------------------------------------------
%	EDUCATION SECTION
%-------------------------------------------------------------------------------
\section{EDUCATION}
\vmarginsmall
\textbf{Moscow Institute Of Physics And Technology} \hfill GPA: 9.51/10; in the top 5\% of the university\\
{\sl Bachelor of Science}, Applied Mathematics \& Computer Science \hfill \textit{08.2016 — 08.2020} \\
Department of Discrete Mathematics\\
\textbf{Relevant courses:} applied statistics, optimization, machine learning, deep learning, stochastic processes, algorithms and data structures (3 terms), concurrent programming, distributed programming
%\textbf{Odessa National University}, Odessa, Ukraine \hfill GPA: 4.93/5 \\
%{\sl Bachelor of Science}, Applied Mathematics \hfill \textit{2014 — 2016}
%-------------------------------------------------------------------------------
%	PROJECTS SECTION
%-------------------------------------------------------------------------------
\section{RESEARCH EXPERIENCE}
\vmarginsmall
\par 
\textbf{Research Internship, Computational Biology, DKFZ/EMBL, Heidelberg,} \textit{01.07.2019 — 01.01.2020}
\textbf{Supervisor:} \href{https://scholar.google.com/citations?user=ClSXZ4IAAAAJ&hl=en}{Oliver Stegle}, PhD (Cambridge), Group Leader at: DKFZ, EMBL Heidelberg, EMBL-EBI\\
My project is about inferring clonal structure in a tumor sample by integrating multiple single-cell modalities (scCNV and scSNV) coming from patients with cancer. I am working on the modifications of the \href{https://github.com/PMBio/cardelino}{Cardelino} model under supervision of \href{https://www.ebi.ac.uk/about/people/yuanhua-huang}{Yuanhua Huang (EMBL-EBI)}. At the same time, I have done a lot of data preparation myself in Python and C++. I tried to write the code capable of fully exploiting the resources of the dedicated HPC cluster. I also learned about the best practices of reproducible research (snakemake pipelines, environment isolation) and applied my knowledge of variational inference. 
\par 
\textbf{Mutual analysis of interaction networks and quantitative trait loci for yeast}, \textit{2018-present}:
\textbf{Supervisor:} \href{https://scholar.google.com/citations?hl=en&user=Arx56RkJBrYC&view_op=list_works&sortby=pubdate}{Yuri Pritykin}, PhD (Princeton), Research Scholar at \href{https://www.mskcc.org/}{MKSCC}\\
Detailed project description is available \href{https://github.com/ivanov-v-v/eQTL_analysis}{on GitHub}.\vspace{0.2em}\\
$ - $ Implemented different approaches to QTL mapping in yeast, from \href{https://www.ncbi.nlm.nih.gov/pubmed/11923494}{basic} to \href{https://elifesciences.org/articles/35471}{state-of-the-art}\\
$ - $ Integrated PPINs into QTL analysis. Implemented statistical tests using \href{http://igraph.org/redirect.html}{igraph} package.\\
$ - $ Carried out GWAS on NGS expression data. Learned how to tackle domain-specific difficulties arising\\ \hphantom{—} from large-scale hypothesis testing using FDR-correction techniques (especially \href{https://github.com/StoreyLab/qvalue}{qvalue}).\\
$ - $ Learned how to write fast and memory-efficient scientific code using numpy, scipy and pandas.\\
$ - $ Practiced parallel programming, interprocess communication and data persistency in Python.\\ 
$ - $ Utilized MIPT supercomputing capabilities, learned how to use SLURM.\\
$ - $ Worked with GeneOntology and KEGG API and related Python/R tools.\vspace{0.2em}
%This project will eventually evolve into my bachelor's thesis.\\
%We plan to submit the paper for publication this spring.
%-------------------------------------------------------------------------------
%	EXPERIENCE SECTION
%-------------------------------------------------------------------------------
% Modify the format of each position
%\begin{format}
%\title{l}\employer{r}\\
%\body\\
%\end{format}
%-------------------------------------------------------------------------------

\section{HONORS}
\vmarginsmall
- \textbf{\href{http://phystech-foundation.org/en/foundation_}{Abramov Scholarship For Academic Excellence}} \hfill{} \textit{2nd term — present}\\
\hphantom{-} Earned by top 10\% students by cumulative GPA in their academic program.\vspace{0.2em}\\
- \textbf{\href{https://studyinrussia.ru/en/study-in-russia/scholarships}{Russian Government Scholarship For International Students}} \hfill{} \textit{2016 — 2020}\\
\hphantom{-} Was selected to become one of 3 Ukrainians to receive the full-coverage scholarship to study CS at the \\\hphantom{-} best Russian universities and got enrolled to MIPT directly, without entrance examination.\vmarginsmall\\
- \textbf{Governor of the Moscow Region Scholarship For Academic Excellence} \hfill \textit{Autumn 2017}\\
\hphantom{-} Awarded termly to excellent students for promising achievements in scientific activities.\vmarginsmall\\
- \textbf{\href{http://winter2019.futurebiotech.ru/}{Future Biotech Winter Retreat "Genome function, editing and therapy"}} \hfill{} \textit{Winter 2019}\\
\hphantom{-}  Became one of 70 young researchers selected to participate in top biotech school in Russia\\ 
\hphantom{-}  sponsored by companies like AstraZeneca, Biocad, GE Healthcare etc. \vmarginsmall\\
- \textbf{\href{https://bioinf.me/en/education}{Summer School in Bioinformatics} by Russian Bioinformatics Institute} \hfill{} \textit{Summer 2017}\\
\hphantom{-} Became one of 50 CS majors selected to participate. Was a member of a hackathon-winning team.\vmarginsmall\\
- \textbf{\href{https://it-edu.mipt.ru/pages/workshops/?lang=en}{Moscow International Workshops in Competitive Programming}} \hfill{} \textit{Autumn 2016, Spring 2017}\\
\hphantom{-}  	Two-times participant of the leading Russian competitive programming bootcamp.\vmarginsmall\\
- \textbf{\href{https://icpc.baylor.edu/}{ACM ICPC} Moscow Subregional Contest (1/4 World Finals)} \hfill{} \textit{Autumn 2017}\\
\hphantom{-} Our team ranked 17 among 301 participating teams and 7th at home university.\vmarginsmall\\
%- \textbf{All-Ukrainian Chemistry Olympiad} — double awardee, triple winner of regional stage\\
%- \textbf{All-Ukrainian Tournament of Young Chemists} — 2nd place
%-------------------------------------------------------------------------------
%	COMPUTER SKILLS SECTION
%-------------------------------------------------------------------------------
%\section{SKILLS}
%\vmargin
%\begin{tabular}{@{}ll}
%	\textbf{Programming languages:} &
%	Python3, R, \LaTeX, Wolfram Mathematica
%	\\
%	& bash, C++ (STL, C++14), C\\
%	\textbf{Bioinformatics-related:} &
%	\textbf{R}: igraph, qvalue, MatrixEQTL, GFLASSO\\
%	& \textbf{Python}: NumPy, SciPy, Pandas, Seaborn, joblib, GEOparse\\
%	\textbf{Other:} 
%	& strong mathematical background, competitive programming experience
%\end{tabular}
%\section{LANGUAGES}
%\vmarginsmall
%English (advanced), German (A2), Russian (native), Ukrainian (native)
\section{REFERENCES} 
\vspace{0.1cm}
\textbf{Dr. Yuri Pritykin} (thesis supervisor), research scholar at MKSCC \hfill\href{mailto://yuri.pritykin@gmail.com}{yuri.pritykin@gmail.com}\\ 
\textbf{Federal Prof. Andrei Raygorodsky}, head of department at MIPT \hfill\href{mailto://mraigor@yandex.ru}{mraigor@yandex.ru}
%-------------------------------------------------------------------------------
%	Interests
%-------------------------------------------------------------------------------
%\section{INTERESTS}
%Computational Genetics, Systems Biology, HPC, Machine Learning, Statstics
%-------------------------------------------------------------------------------
\end{resume}
\end{document}